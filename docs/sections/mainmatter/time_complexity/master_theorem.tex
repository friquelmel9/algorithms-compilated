\section{Teorema maestro}
Para determinar la complejidad de un algoritmo recursivo se utiliza
a menudo el \textbf{teorema maestro}.\\

Formalmente, se define la siguiente relación para casos grandes:

\begin{center}
  \(T(n) \leq a \cdot T\left(\frac{n}{b}\right) + O(n^{d})\)\\
  \begin{figure}[h]
    Parámetros1:\\
    $a$ es el número de llamadas recursivas.\\
    $b$ el factor de disminución del tamaño de entrada (sub-problemas).\\
    $d$ el exponente asociado al trabajo no recursivo.
  \end{figure}
\end{center}

De esta relación y si se cumple que $a \geq 1, b > 1 y d \geq 0$, se cumple que:

\begin{center}
  \[
    T(n) =
    \begin{cases}
      O(n^{d} \log n) & \text{si } a = b^{d} \\
      O(n^{d}) & \text{si } a < b^{d} \\
      O(n^{\log_{b} a}) & \text{si } a > b^{d}
    \end{cases}
  \]
\end{center}
