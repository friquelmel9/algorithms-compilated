\section{Desempeño de algoritmos}
Los algoritmos no tienen un rendimiento fijo para cualquier tipo de
entrada. Generalmente, hay situaciones donde se comportan de mejor
o peor manera dependiendo de la noción de la cual trabajan. En el
análisis y diseño de algoritmos se tienen en cuenta los siguientes
casos para evaluar el rendimiento de un algoritmo.

\subsection{Mejor caso}
Caso donde el algoritmo funciona de la mejor manera. Por ejemplo,
el mejor caso de algunos algoritmos de ordenamiento se considera
el que la entrada ya esté ordenada.

\subsection{Caso promedio}
Caso donde el algoritmo posee un tiempo de ejecución esperado
dentro una distribución de entradas. Para comprobarlo, se necesita
una hipótesis probabilística.

\subsection{Peor caso}
Caso donde el algoritmo trabaja de peor forma. Es la situación que
se utiliza generalmente para las notaciones asíntoticas. Un ejemplo
es una entrada inversamente ordenada para los algoritmos de
ordenamiento.

\subsection{Caso amortizado}
Costo promedio de \textbf{una operación} en una secuencia larga de
operaciones que no depende de la distribución de la entrada.
Un ejemplo es el método \textit{push\_back} para la clase \textit{vector}
de \textit{C++}.
