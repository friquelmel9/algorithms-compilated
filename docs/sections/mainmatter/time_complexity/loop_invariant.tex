\section{Invariante de ciclo}
Una técnica que se utiliza a menudo para demostrar la correctitud de un
algoritmo es usar un \textbf{invariante de ciclo.} Un \textit{invariante}
es una afirmación o predicado sobre un conjunto de variables relacionadas
que es \textbf{siempre verdadero}. La idea es que este invariante sea
\textbf{constante} en cada iteración y que explique la relación entre
las variables de entrada y salida del ciclo.

\subsection{Etapas del invariante de ciclo}
\textbf{Inicialización: }El invariante debe ser verdadero
\textbf{antes de empezar} el ciclo (caso base).\\

\textbf{Mantención: }El invariante es verdadero
\textbf{antes de cada iteración del ciclo} y se mantiene
verdadero hasta \textbf{antes de la siguiente iteración.}
A menudo se utiliza \textbf{inducción} para probar este
punto con una \textit{hipótesis inductiva} para luego
realizar un \textit{paso inductivo}.\\

\textbf{Finalización: }El invariante debe ser verdadero
\textbf{después de terminar} el ciclo.\\

\textbf{Conclusión: }Si el invariante es verdadero en cada etapa, el
algoritmo es correcto.
