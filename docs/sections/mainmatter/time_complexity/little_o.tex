\subsection{Notación little o}
La notación \textit{little o} se relaciona con $T(n)$.\\

A diferencia de \textit{Big O}, la notación \textit{little o} permite
afirmar que $T(n)$ \textbf{siempre} es menor que $c\cdot f(n)$ por muy
pequeño que sea $c$ a partir de un $n_{0}$ y puede hacerse más pequeña
de manera arbitraria.\\

Formalmente, se define como:

\begin{center}
  \(T(n) = o(f(n))\hspace{0.2cm}ssi\hspace{0.2cm}\exists c, n_{0} > 0\hspace{0.2cm}t.q\hspace{0.2cm}T(n) < c\cdot f(n), \forall{n} \geq n_{0}\)
\end{center}

Por otro lado, también se puede intuir a través de esta relación:

\begin{center}
  \[
    \lim_{n\to\infty} \left(\frac{T(n)}{f(n)}\right) = 0
  \]
\end{center}
