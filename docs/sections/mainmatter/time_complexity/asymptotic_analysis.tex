\section{Análisis asintótico}
Una manera de medir la complejidad de un algoritmo es a través
del \textit{análisis asintótico}. Dentro de las dimensiones
estudiadas, se consideran dos grandes aspectos:

\begin{itemize}
  \item{Complejidad temporal ($T(n)$): Predicción de operaciones
        fundamentales en base al tamaño de la entrada.}
  \item{Complejidad espacial ($E(n)$): Predicción del costo en
        memoria en base al tamaño de la entrada.}
\end{itemize}

En base a estos criterios, se definen las siguientes notaciones:
\subsection{Notación Big O}
La notación \textit{Big O} se relaciona con $T(n)$. \\

Se utiliza $O(f(n))$ como representación del límite en el peor caso de
tiempo de ejecución de un algoritmo en base a un tamaño de entrada $n$.\\

Formalmente, se define como:

\begin{center}
  \(T(n) = O(f(n))\hspace{0.2cm}ssi\hspace{0.2cm}\exists c, n_{0} > 0\hspace{0.2cm}t.q\hspace{0.2cm}T(n) \leq c\cdot f(n), \forall{n} \geq n_{0}\)
\end{center}

\subsection{Notación little o}
La notación \textit{little o} se relaciona con $T(n)$.\\

A diferencia de \textit{Big O}, la notación \textit{little o} permite
afirmar que $T(n)$ \textbf{siempre} es menor que $c\cdot f(n)$ por muy
pequeño que sea $c$ a partir de un $n_{0}$ y puede hacerse más pequeña
de manera arbitraria.\\

Formalmente, se define como:

\begin{center}
  \(T(n) = o(f(n))\hspace{0.2cm}ssi\hspace{0.2cm}\exists c, n_{0} > 0\hspace{0.2cm}t.q\hspace{0.2cm}T(n) < c\cdot f(n), \forall{n} \geq n_{0}\)
\end{center}

Por otro lado, también se puede intuir a través de esta relación:

\begin{center}
  \[
    \lim_{n\to\infty} \left(\frac{T(n)}{f(n)}\right) = 0
  \]
\end{center}

\subsection{Notación Big Omega}
La notación \textit{Big Omega} se relaciona con $T(n)$.\\

Se utiliza $\Omega(f(n))$ como representación de la cota inferior del tiempo de
ejecución de un algoritmo para un tamaño de entrada $n$.\\

Formalmente, se define como:

\begin{center}
  \(T(n) = \Omega(f(n))\hspace{0.2cm}ssi\hspace{0.2cm}\exists c, n_{0} > 0\hspace{0.2cm}t.q\hspace{0.2cm}T(n) \geq c\cdot f(n), \forall{n} \geq n_{0}\)
\end{center}

\subsection{Notación little omega}
La notación \textit{litte omega} se relaciona con $T(n)$.\\

Similar a la notación \textit{little o}, la notación \textit{little omega}
permite afirmar que $T(n)$ es \textbf{estrictamente} mayor que $c\cdot f(n)$
a partir de un $n_{0}$ sea cual sea el valor de $c$.\\

Formalmente, se define como:

\begin{center}
  \(T(n) = \omega(f(n))\hspace{0.2cm}ssi\hspace{0.2cm}\exists c, n_{0} > 0\hspace{0.2cm}t.q\hspace{0.2cm}T(n) > c\cdot f(n), \forall{n} \geq n_{0}\)
\end{center}

Paralelamente, se puede intuir a partir de la siguiente relación:

\begin{center}
  \[
    \lim_{n\to\infty} \left(\frac{T(n)}{f(n)}\right) = \infty
  \]
\end{center}

\subsection{Notación Big Theta}
La notación \textit{Big Theta} se relaciona con $T(n)$.\\

Se utiliza $\Theta(f(n))$ para un $T(n)$ donde se cumple que
\(T(n) = \O(f(n))\) y \(T(n) = \Omega(f(n))\) para simbolizar
la cota superior e inferior en tiempo de ejecución de un
algoritmo frente a un tamaño de entrada $n$.\\

Formalmente, se define como:

\begin{center}
  \(T(n) = \Theta(f(n))\hspace{0.2cm}ssi\hspace{0.2cm}\exists c_{1}, c_{2}, n_{0} > 0\hspace{0.2cm}t.q\hspace{0.2cm}c_{1}\cdot f(n) \leq T(n) \leq c_{2}\cdot f(n), \forall{n} \geq n_{0}\)
\end{center}

