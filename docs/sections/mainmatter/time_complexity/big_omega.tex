\subsection{Notación Big Omega}
La notación \textit{Big Omega} se relaciona con $T(n)$.\\

Se utiliza $\Omega(f(n))$ como representación de la cota inferior del tiempo de
ejecución de un algoritmo para un tamaño de entrada $n$.\\

Formalmente, se define como:

\begin{center}
  \(T(n) = \Omega(f(n))\hspace{0.2cm}ssi\hspace{0.2cm}\exists c, n_{0} > 0\hspace{0.2cm}t.q\hspace{0.2cm}T(n) \geq c\cdot f(n), \forall{n} \geq n_{0}\)
\end{center}
