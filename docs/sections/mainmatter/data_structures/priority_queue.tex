\section{Priority Queue}

Una priority queue permite mantener los datos de mayor prioridad con acceso $O(1)$.
Esta mayor prioridad generalmente se basa en base al valor.

\begin{cppcode}
    /**
    * Crea una priority queue del tipo T
    * El elemento al tope de la priority queue sera el de mayor valor
    */
    std::priority_queue<T> pq;

    /**
    * Crea una priorty queue del tipo T
    * El elemento al tope de la priority queue sera aquel de menor valor
    */
    std::priority_queue<T, std::vector<T>, std::greater<T>> min_pq;

    /**
    * Retorna el tamaño de la priority queue
    * @timecomplexity: O(1)
    */
    pq.size();

    /**
    * Inserta un elemento en la priority queue
    * @timecomplexity: O(\log{n})
    */
    pq.insert(item);

    /**
    * Elimina el elemento de mayor prioridad
    * @timecomplexity: O(\log{n})
    */
    pq.pop();

    /**
    * Obtiene el elemento al tope de la priority queue
    * @timecomplexity: O(1)
    */
    pq.top();
\end{cppcode}