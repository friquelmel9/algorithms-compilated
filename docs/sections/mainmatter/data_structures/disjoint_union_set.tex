\section{Disjoint Union Set}

El disjointed union set permite identificar si dos elementos estan del mismo set.
Si no entiendes su funcionamiento, no pasa nada. Utiliza la siguiente implementación para algoritmos como Kruskal

\begin{cppcode}
    
    /**
    * Clase que implementa un Disjoint Set Union (DSU)
    */
    class DisjointSetUnion{

    private:

        std::vector<int> parent;
        std::vector<int> rank; // Para mantener la altura del arbol

    public:

        /**
        * 
        * @time_complexity: O(n)
        * @space_complexity: O(n)
        */
        DisjointSetUnion(int n){
            parent.resize(n);
            std::iota(parent.begin(), parent.end(), 0);
            rank.assign(n, 0);
        }
        
        /**
        * 
        * @time_complexity: O(\alpha(N))*
        * @space_complexity: O(n)
        */
        int find(int i) {
            if (parent[i] == i) {return i; }
            return parent[i] = find(parent[i]);
        }
        
        /**
        * 
        * @time_complexity: O(\alpha(N))*
        * @space_complexity: O(n)
        */
        bool unite(int x, int y){
            int root_x = find(x);
            int root_y = find(y);
            
            if(root_x != root_y){
                
                if (rank[root_x] < rank[root_y]) {
                parent[root_x] = root_y;
                } else if (rank[root_x] > rank[root_y]) {
                    parent[root_y] = root_x;
                } else {
                    parent[root_y] = root_x;
                    rank[root_x]++; 
                }
                return true;
                
            }
            
            return false;
        }

    };
    
\end{cppcode}