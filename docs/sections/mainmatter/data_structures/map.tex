\section{Map}

Un map es una estructura de datos que guarda valores en la forma llave : valor.

\begin{cppcode}
    
    /**
    * Declaración de un map con llave del tipo T1 y valor del tipo T2
    */
    std::map<T1,T2> map;

    /**
    * Retorna el tamaño del map
    * @timecomplexity: O(1)
    */
    map.size();

    /**
    * Inserta un par llave - valor dentro del map
    * @timecomplexity: O(\log{n})
    */
    map.insert({llave,valor});

    /**
    * Elimina el par que contenga la llave
    * @timecomplexity: O(\log{n})
    */
    map.erase(llave);

    /**
    * Retorna un iterador hacia item. Retorna un iterador al final si no existe
    * @timecomplexity: O(\log{n})
    * 
    */
    itr = map.find(llave);

    /**
    * Acceso al primer elemento (llave)
    */
    itr -> first;

    /**
    * Acceso al segundo elemento (valor)
    */
    itr -> second;
    
\end{cppcode}

\subsection{Unordered Map}

El unordered map cumple con las mismas propiedades que un Map.
Su complejidades pasan a ser $O(1)$, dado de pasar de un Red-Black tree a un hash

\begin{cppcode}
    /**
    * Declaración de un unordered_map con llave del tipo T1 y valor del tipo T2
    */
    std::unordered_map<T1,T2> unordered_map;

    /**
    * Retorna el tamaño del unordered_map
    * @timecomplexity: O(1)
    */
    unordered_map.size();

    /**
    * Inserta un par llave - valor dentro del unordered_map
    * @timecomplexity: O(1)
    */
    unordered_map.insert({llave,valor});

    /**
    * Elimina el par que contenga la llave
    * @timecomplexity: O(1)
    */
    unordered_map.erase(llave);

    /**
    * Retorna un iterador hacia item. Retorna un iterador al final si no existe
    * @timecomplexity: O(1)
    * 
    */
    itr = unordered_map.find(llave);

    /**
    * Acceso al primer elemento (llave)
    */
    itr -> first;

    /**
    * Acceso al segundo elemento (valor)
    */
    itr -> second;
\end{cppcode}