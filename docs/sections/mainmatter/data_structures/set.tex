\section{Set}

Un set es una estructura de datos que permite guardar elementos unicos.

\begin{cppcode}
    /**
    * Declaración de un set de tipo T
    */
    std::set<T> set;

    /**
    * Retorna el tamaño del set
    * @timecomplexity: O(1)
    */
    set.size();

    /**
    * Inserta un elemento al set
    * @timecomplexity: O(\log{n})
    */
    set.insert(item);

    /**
    * Elimina el elemento del set
    * @timecomplexity: O()
    */
    set.erase(item);

    /**
    * Retorna un iterador al elemento item. En caso que no existe retorna un iterador al final
    * @timecomplexity: O(\log{n})
    */
    set.find(item);

    /**
    * Recorrer el set con un iterador. Permite obtener los elementos en orden basado en red-black-tree
    * @timecomplexity: O(n)
    */
    for(auto itr = set.begin(); itr != set.end(); itr++){}
\end{cppcode}

\subsection{Unordered Set}

Un unordered set cumple con las mismas propiedades que el set, pero con una complejidad distinta.
Su complejidad pasa a ser $O(1)$ dada a pasar de un Red-Black Tree a funciones hash.

\begin{cppcode}
    
    /**
    * Declaración de un unordered_set de tipo T
    */
    std::unordered_set<T> unordered_set;

    /**
    * Retorna el tamaño del unordered_set
    * @timecomplexity: O(1)
    */
    unordered_set.size();

    /**
    * Inserta un elemento al unordered_set
    * @timecomplexity: O(1)
    */
    unordered_set.insert(item);

    /**
    * Elimina el elemento del unordered_set
    * @timecomplexity: O(1)
    */
    unordered_set.erase(item);

    /**
    * Retorna un iterador al elemento item. En caso que no existe retorna un iterador al final
    * @timecomplexity: O(1)
    */
    unordered_set.find(item);
    
\end{cppcode}