\section{Fibonnaci}

The Fibonacci numbers, commonly denoted F(n) form a sequence, called the Fibonacci sequence, such that each number is the 
sum of the two preceding ones, starting from 0 and 1. That is,

$$F(0) = 0, F(1) = 1$$ 
$$F(n) = F(n - 1) + F(n - 2), \text{for} n > 1.$$

Given n, calculate F(n).

Example 1:

Input: n = 2 \\
Output: 1 \\
Explanation: $F(2) = F(1) + F(0) = 1 + 0 = 1$.

Example 2:

Input: n = 3\\
Output: 2\\
Explanation: $F(3) = F(2) + F(1) = 1 + 1 = 2$.

Example 3:

Input: n = 4\\
Output: 3\\
Explanation: $F(4) = F(3) + F(2) = 2 + 1 = 3$.

Constraints:

$0 <= n <= 30$

\begin{cppcode}
    class Solution \{
        public:
            std::map<int, int> dp;
            
            Solution()\{
                dp[0] = 0;
                dp[1] = 1;
            \}

            /**
            * Calculates fibonacci sequence using a Top-Down approach
            * - Base case: 0 and 1 results added previously on memory. Memory[0] = 0 and Memory[1] = 1
            * 
            * @param number Number of the fibonacci sequence to be calculated.
            * @param memory Map that stores previos calculated numbers in fibonnaci
            * 
            */
            int fib(int n) \{
                
                if(dp.find(n) == dp.end()){ dp[n] = fib(n-1) + fib(n-2);}
                return dp[n];

            \}
            
        \};
\end{cppcode}

